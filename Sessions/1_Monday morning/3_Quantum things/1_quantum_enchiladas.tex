
\documentclass[a4paper, 11pt]{article}
\usepackage[paper=a4paper,width=15.5cm,height=20cm,top=3cm,left=3cm]{geometry}
\usepackage[T1]{fontenc}
\usepackage[utf8]{inputenc}
\usepackage{mathtools, amssymb, latexsym}

% Comandos para el abstract.
\newcommand{\abstracttitle}[1]{{
    \centering
    \LARGE \textbf{#1}\\
    \vspace*{0.7cm}
}}
\newcommand{\firstauthor}[2]{{
    \centering
    \underline{#1}, \textsf{#2}\\
    \vspace*{0.25cm}
}}
\newcommand{\otherauthor}[2]{{
    \centering
    #1, \textsf{#2}\\
    \vspace*{0.25cm}
}}
\newcommand{\abstracttext}[1]{
    \vspace{0.6cm}
    #1
}


\begin{document}

\abstracttitle{On the quantum equilibrium of quantum lasagnas}

\firstauthor{Wendy S. Robinson}{University of Andorra}
\otherauthor{Michael M. Williams}{University of Andorra}

\abstracttext{
    Since it was discovered in 1788, the quantum lasagna effect has cautivated all sorts of scientists. Being a relatively new field of study, still do not has a lot of attention from the community and many questions remain unsolved: how does this effect has to say about the salad conjecture on quantum termonutrition? How much quantumness is necessary to turn a meat lasagna into a vegetarian lasagna? Do quantum lasagnas really exist?

    In this talk, we offer some answers to this questions\footnote{Whether the answers are true or false, we do not address that issue.}. Moreover, we introduce the concept of quantum enchilada, which we think it could help solving the salad conjecture by means of the following formula:
    \begin{align}
        A = b.
    \end{align}

    This work has been partially funded by the Quantum Institute of even Quantumer Food.
}

\begin{thebibliography}{9}
\bibitem{quantum1}
L. Nielsen and V. Kilmer
\newblock Quantum rise of the quantum food
\newblock {\em Quantum newsletter}, 67(4):3--15, 2017.
\end{thebibliography}


\end{document}
