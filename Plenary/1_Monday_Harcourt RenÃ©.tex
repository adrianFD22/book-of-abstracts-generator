
\documentclass[a4paper, 11pt]{article}
\usepackage[paper=a4paper,width=15.5cm,height=20cm,top=3cm,left=3cm]{geometry}
\usepackage[T1]{fontenc}
\usepackage[utf8]{inputenc}
\usepackage{mathtools, amssymb, latexsym, tikz}

% Comandos para el abstract.
\newcommand{\abstracttitle}[1]{{
    \centering
    \LARGE \textbf{#1}\\
    \vspace*{0.7cm}
}}
\newcommand{\firstauthor}[2]{{
    \centering
    \underline{#1}, \textsf{#2}\\
    \vspace*{0.25cm}
}}
\newcommand{\otherauthor}[2]{{
    \centering
    #1, \textsf{#2}\\
    \vspace*{0.25cm}
}}
\newcommand{\abstracttext}[1]{
    \vspace{0.6cm}
    #1
}


\begin{document}

\abstracttitle{Solving some millennium problems}

\firstauthor{Harcourt René}{University of Harvard}

\abstracttext{
    I solve every millenium problem only using the pidgeonhole principle: Riemann conjecture, Navier-Stokes equations... You named them. As a corollary to all them together I obtain the Ramsey number $R(2878,29899)$.
}

\begin{thebibliography}{9}
\bibitem{harcourt}
Tânia Almeida Araujo
\newblock Some counterexamples to millenium problems
\newblock {\em $\omega$th TIAFC}: 8, 2024
\end{thebibliography}


\end{document}
