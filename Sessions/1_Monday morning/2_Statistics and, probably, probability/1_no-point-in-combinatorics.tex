
\documentclass[a4paper, 11pt]{article}
\usepackage[paper=a4paper,width=15.5cm,height=20cm,top=3cm,left=3cm]{geometry}
\usepackage[T1]{fontenc}
\usepackage[utf8]{inputenc}
\usepackage{mathtools, amssymb, latexsym}

% Comandos para el abstract.
\newcommand{\abstracttitle}[1]{{
    \centering
    \LARGE \textbf{#1}\\
    \vspace*{0.7cm}
}}
\newcommand{\firstauthor}[2]{{
    \centering
    \underline{#1}, \textsf{#2}\\
    \vspace*{0.25cm}
}}
\newcommand{\otherauthor}[2]{{
    \centering
    #1, \textsf{#2}\\
    \vspace*{0.25cm}
}}
\newcommand{\abstracttext}[1]{
    \vspace{0.6cm}
    #1
}


\begin{document}

\abstracttitle{Explaining combinatorics via probability theory}

\firstauthor{Yvette Dolgorukova}{University of North Carolina}
\otherauthor{Natalie Yefimova}{University of North Carolina}

\abstracttext{
    Since time immemorial, this two fields has been deeply entangled. There numerous connections between combinatorics and probability that induce one to think about if there is a deeper relation which we are missing. In this talk, we respond to this uncertainty making use of the following fundamental equality obtained in \cite{kolmogorov}:
    \begin{equation} \label{equation:useless}
        \text{Combinatorics} = \frac{1}{\text{Probability}}
    \end{equation}
    Since combinatorics can be easily derived from probability theory using (\ref{equation:useless}), we conclude there is no point in studying that field.
}

\begin{thebibliography}{9}
\bibitem{kolmogorov}
Kolmogorov
\newblock On probable things that could probably happen
\newblock {\em Journal of probablitlyliyt}: 3-100, 1633
\end{thebibliography}


\end{document}
